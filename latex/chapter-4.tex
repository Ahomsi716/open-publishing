\documentclass{article}

\usepackage{caption}
                
\usepackage[backend=biber,hyperref=false,citestyle=authoryear,bibstyle=authoryear]{biblatex}
                
\bibliography{bibliography}
            
\usepackage{graphicx}
                
\usepackage{calc}
                
\newlength{\imgwidth}
                
\newcommand\scaledgraphics[2]{%
                
\settowidth{\imgwidth}{\includegraphics{#1}}%
                
\setlength{\imgwidth}{\minof{\imgwidth}{#2\textwidth}}%
                
\includegraphics[width=\imgwidth,height=\textheight,keepaspectratio]{#1}%
                
}
            
\begin{document}

\title{Wikidata - erste Schritte}

\maketitle





[erste Schritte, einfache Sachen, Arbeiten mit]

\begin{figure}
\scaledgraphics{c974b28d-908a-42e2-9445-5efb50e2817b.png}{1}
\caption*{"lebendes" Handbuch Wikidata}\label{F44920591}
\end{figure}





Im Handbuch Wikidata \autocite{solohub_handbuch_2021} geben Studierende eine Einführung in ...


Die WikidataTours (...) sind vor allem


\printbibliography[title={Literaturverzeichnis}]
\end{document}
