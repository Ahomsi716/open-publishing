\documentclass{article}

                
\usepackage[backend=biber,hyperref=false,citestyle=authoryear,bibstyle=authoryear]{biblatex}
                
\bibliography{bibliography}
            
\begin{document}

\title{Wikidata und die OpenGLAM Community}

\maketitle


Open Glam bezeichnet ein Netzwerk aus Galerien, Bibliotheken, Archiven und Museen und setzt sich für die gemeinsame Nutzung kulturellen Erbes ein. Das Wort GLAM setzt sich dabei uns den englischen Wörtern „Galleries, Libraries, Archives, Museums“ zusammen. \autocite{FidusWriter}


\emph{[Inhalte könnten sein: beispielhafte Projekte oder Werkzeuge // Fokus auf visuelle Inhalte? // Aktivitäten und/oder Ressourcen der Wikimedia Community OpenGLAM]}


Die OpenGLAM Working Group innerhalb von Opendata.ch, dem schweizer Chapter der Open Knowledge Foundation Initiative richtet jählich einen OpenGLAM Hackathon aus (https://glam.opendata.ch/hackathons/ ). Zielgruppe sind (übersetzen: data providers, software developers, digital humanists, artists, Wikimedians/Wikipedians, and other interested people in order to experiment how cultural data and content can be used for research purposes, for web and mobile apps, in the context of Wikipedia, for artistic re-mixes, or for other forms of re-use.)  Die Verasntaltungsserie dient einerseits dazu..., .... Neu im Programm sind spezielle Angebote zu "Wikidata for GLAM" als kostenlose ... (übersetzen und ergänzen: free introductory courses to WikiData for GLAM institutions, schauen Sie mal hier: https://glam.opendata.ch/wikidata-for-glam/)


Auch die Open Knowledge Foundation hatte bereits 2017 mit Einführungskursen gestartet https://okfn.de/blog/2017/05/wikidata-f\%C3\%BCr-openglam/. Mittlerweile ...


Wichtigste Gruppe innerhalb der schon aktiven Wikidata-Community mit dem Schwerpunkt GLAM ist hier aktiv https://www.wikidata.org/wiki/Wikidata:WikiProject\_Cultural\_heritage


\printbibliography[title={Literaturverzeichnis}]
\end{document}
