\documentclass{article}

\usepackage{hyperref}
\usepackage{caption}
\usepackage{wrapfig}
                
\usepackage[backend=biber,hyperref=false,citestyle=authoryear,bibstyle=authoryear]{biblatex}
                
\bibliography{bibliography}
            
\usepackage{graphicx}
                
\usepackage{calc}
                
\newlength{\imgwidth}
                
\newcommand\scaledgraphics[2]{%
                
\settowidth{\imgwidth}{\includegraphics{#1}}%
                
\setlength{\imgwidth}{\minof{\imgwidth}{#2\textwidth}}%
                
\includegraphics[width=\imgwidth,height=\textheight,keepaspectratio]{#1}%
                
}
            
\begin{document}

\title{Wikidata und die OpenGLAM Community}

\maketitle




\begin{wrapfigure}{l}{0.5\textwidth}
\scaledgraphics{ea614bab-3f4c-4c33-9fda-586c650b58c5.png}{0.5}
\caption*{Foto 1}\label{F76287561}
\end{wrapfigure}

\begin{wrapfigure}{r}{0.5\textwidth}
\scaledgraphics{ce15208f-6cbe-4d4f-a6eb-ffd2e7f8fb24.png}{0.5}
\caption*{Foto 2}\label{F81154591}
\end{wrapfigure}




















Open Glam bezeichnet ein Netzwerk aus Galerien, Bibliotheken, Archiven und Museen und setzt sich für die gemeinsame Nutzung kulturellen Erbes ein. Das Wort GLAM setzt sich dabei uns den englischen Wörtern „Galleries, Libraries, Archives, Museums“ zusammen. \autocite{FidusWriter}


\emph{[Inhalte könnten sein: beispielhafte Projekte oder Werkzeuge // Fokus auf visuelle Inhalte? // Aktivitäten und/oder Ressourcen der Wikimedia Community OpenGLAM]}


Die OpenGLAM Working Group innerhalb von Opendata.ch, dem schweizer Chapter der Open Knowledge Foundation Initiative richtet jählich einen OpenGLAM Hackathon aus (https://glam.opendata.ch/hackathons/ ). Zielgruppe sind (übersetzen: data providers, software developers, digital humanists, artists, Wikimedians/Wikipedians, and other interested people in order to experiment how cultural data and content can be used for research purposes, for web and mobile apps, in the context of Wikipedia, for artistic re-mixes, or for other forms of re-use.)  Die Verasntaltungsserie dient einerseits dazu..., .... Neu im Programm sind spezielle Angebote zu "Wikidata for GLAM" als kostenlose ... (übersetzen und ergänzen: free introductory courses to WikiData for GLAM institutions, schauen Sie mal hier: https://glam.opendata.ch/wikidata-for-glam/)


Auch die Open Knowledge Foundation hatte bereits 2017 mit Einführungskursen gestartet https://okfn.de/blog/2017/05/wikidata-f\%C3\%BCr-openglam/. Mittlerweile ...


Innerhalb der bereits aktiven Wikidata-Community hat sich die Gruppe unter https://www.wikidata.org/wiki/Wikidata:WikiProject\_Cultural\_heritage zum Ziel gesetzt, "Wikidata als zentralen Hub für Datenintegration, Datenanreicherung und Datenmanagement im Bereich des Kulturerbes zu etablieren". Wikidata. „Wikidata:WikiProject Cultural Heritage – Wikidata“. Zugegriffen 2. Juni 2022. \href{https://www.wikidata.org/wiki/Wikidata:WikiProject_Cultural_heritage}{https://www.wikidata.org/wiki/Wikidata:WikiProject\_Cultural\_heritage}.


Mehr (auch für die anderen Kapitel): auf der o.g. Wikiprojekts-Seite unter Tutorials, Guidelines, Blog Posts 


\printbibliography[title={Literaturverzeichnis}]
\end{document}
